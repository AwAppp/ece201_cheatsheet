\documentclass{report}

\title{ECE 205A midterm cheat sheet}
\author{AnPing Tao}
\date{\today}

\usepackage{amsmath}
\usepackage{amssymb}
\usepackage{amsthm}
\usepackage{xcolor}
\usepackage{color}
\usepackage{cancel}
\usepackage{pdfpages}
\usepackage{multirow}
\usepackage{rotating}

\def\changemargin{\list{}{\rightmargin 0.5cm \leftmargin 0.5cm}\item[]}
\let\endchangemargin=\endlist

\renewcommand{\qedsymbol}{$\blacksquare$}

\setlength{\parindent}{0pt}

% \definecolor{mybg}{RGB}{72, 38, 82}
% \definecolor{mytext}{RGB}{177, 215, 183}

% \color{mytext}
% \pagecolor{mybg}

\begin{document}

\textbf{Def:} A \underline{vector} is a tuple of numbers.

\textbf{Def:} The \underline{dimension} is the size of this tuple.

\textbf{Def:} A \underline{matrix} is a 2-dimensional grid of numbers.

\textbf{Def:} $\mathbb{R}^{m \times n}$ denotes all $m \times n$ matrices with field $\mathbb{R}$.

\textbf{Def:} $\mathbb{C}^{m \times n}$ denotes all $m \times n$ matrices with field $\mathbb{C}$.

\textbf{Def:} vector-matrix product:
$\begin{bmatrix}
        a & b \\
        c & d
    \end{bmatrix} \cdot
    \begin{bmatrix}
        x \\y
    \end{bmatrix} =
    \begin{bmatrix}
        ax + by \\
        cx + dy
    \end{bmatrix}$
(Inner dimensions must agree. Outer dimensions remain.)

$\begin{bmatrix}
        ax + by \\
        cx + dy
    \end{bmatrix} =
    x \begin{bmatrix}
        a \\c
    \end{bmatrix} + y
    \begin{bmatrix}
        b \\d
    \end{bmatrix}$: Express matrix-vector multiplication as a linear combination of the matrix.

\textbf{Def:} \underline{Inner product}
$<\begin{bmatrix}
        x_1 \\ x_2 \\ \vdots \\ x_n
    \end{bmatrix},
    \begin{bmatrix}
        y_1 \\ y_2 \\ \vdots \\ y_n
    \end{bmatrix}> =
    \begin{bmatrix}
        x_1 & x_2 & \cdots & x_n
    \end{bmatrix}
    \begin{bmatrix}
        y_1 \\ y_2 \\ \vdots \\ y_n
    \end{bmatrix} = x_1y_1 + x_2y_2 + \cdots + x_ny_n$

\textbf{Def:} \underline{Matrix Multiplication:} $A \in \mathbb{R}^{m \times k}, B \in \mathbb{R}^{k \times n}$,
$C = AB \in \mathbb{R}^{m \times n}$ and $C_{ij} = \sum_{k=1}^{n} A_{ik}B_{kj}$.

\textbf{Def:} \underline{span} of vectors $v_1, v_2, \dots, v_n$ is the set of vectors that can be obtained as
linear combination of the vectors $v_1, v_2, \dots, v_n$:

$span\{v_1, v_2, \dots, v_n\} = a_1v_1 + a_2v_2 + \dots + a_nv_n$ for $a_1, a_2, \dots, a_n$ are scalars.
\end{document}